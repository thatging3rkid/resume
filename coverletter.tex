% coverletter.tex
%
% A cover letter for my resume
%
% @author Connor Henley, @thatging3rkid
% @author Chris Culpepper, @cculpapper
% @note see res.cls for more authors

% LaTeX resume using res.cls
\documentclass[line,margin]{res} 
%\usepackage{helvetica} % uses helvetica postscript font (download helvetica.sty)
%\usepackage{newcent}   % uses new century schoolbook postscript font 
\usepackage{multicol}
%\usepackage[top=.5in, bottom=.5in, left=.5in, right=.5in]{geometry}
\usepackage{textcomp}
\usepackage{chngpage} % rip changepage?

\begin{document}
\setlength\columnsep{-30pt}
%\setlength{\multicolsep}{6.0pt plus 2.0pt minus 1.5pt}% 50% of original values
\name{Connor F. Henley}
\email{} % school email
\phone{} % personal email
\address{} % address line one
\address{} % address line two
\website{https://connorhenley.engineer}
\website{} % phone number

\begin{resume}
 \setlength
 \multicolsep{2pt}
 \begin{adjustwidth}{-92pt}{0pt}
 \vspace{10pt}

\noindent
Dear Hiring Professional, \\

\noindent
I am a second year student at the Rochester Institute of Technology in the Computer Engineering Dual Degree (BS/MS) program. I am seeking a co-op position in the computer engineering field and I am available May 15th through December 2018. \\

\noindent
Through my coursework and personal projects, I found that I have a strong programming background, especially in C and Java. I have taken multiple programming courses, like the Mechanics of Programming, which is an application programming course using C and Linux, and Introduction to Software Engineering, where we learn a modern software development process (Agile and Scrum), and then implement the process with a group of peers through the WebCheckers project. I have taken or are taking classes like Digital System Design I \& II, where I learned about digital electronics and hardware, Circuits I and Computer Organization to gain experience working with computer hardware. \\

\noindent
Outside of class, I have worked on many personal projects that have grown my background. One of my projects was building a 3D printer from scratch, where I learned a lot about software and hardware interfacing and working with embedded systems. Starting in middle school, I have been involved in the FIRST robotics program, which is a program for encouraging people to get interested in science, technology, engineering and math through competitive robotics. I joined the FIRST Lego League (FLL) team in 6th grade and went to bring my team to the state championship for the first time. I then went on to the FIRST Robotics Competition team in high school, where I led the controls sub-team and was the co-captain of the team my senior year. I also worked on extra projects for my FRC team, for example, I wrote a library for a swerve drive-train. In FRC, a swerve drive-train uses four high-traction wheels (one in each corner) that can be rotated to the optimal angle, allowing the robot to go in any direction without the traction loss that other solutions provide. On this project, I learned a lot about the software design process, good documentation, and using version control. Additionally, I have continued being involved in the FIRST community by joining the RIT FIRST club, where we mentor local teams and volunteer at events. Additionally, we work on a project for the Imagine RIT creativity and innovation festival, where we build a scaled down version of FRC from the ground up, where I have learned a lot about debugging a project, improved my teamwork skills, and learned about simple analog electronics. \\

\noindent
However, many jump to the conclusion that my time in FIRST robotics has only taught me technical skills, when in reality, teamwork is one of the most important aspects of the FIRST robotics program. And during my time as a student in the program, I spent a lot of team in leadership positions. As I mentioned above, I was the lead of the controls team for three years, and in my last year, I led the team to win the Innovation in Controls Award. Additionally, I was elected to the co-captain position my senior year, and led the team to the district championship for the first time since it's introduction in 2014. These positions also taught me a lot about how to divide work, manage people, and communicate better in a small group. Using the skills I gained in FIRST robotics, I found myself leading group projects in the RIT community. \\

\noindent
I believe that I am qualified for your co-op position and I know that I would be a valuable asset to your organization. I would appreciate the opportunity to discuss my qualifications in more detail. \\

Sincerely, \\
Connor F. Henley

\end{adjustwidth}
\end{resume}
\end{document}