% resume.tex
%
% My resume
%
% @author Connor Henley, @thatging3rkid
% @author Chris Culpepper, @cculpapper
% @note see res.cls for more authors

% LaTeX resume using res.cls
\documentclass[line,margin]{res} 
%\usepackage{helvetica} % uses helvetica postscript font (download helvetica.sty)
%\usepackage{newcent}   % uses new century schoolbook postscript font 
\usepackage{multicol}
%\usepackage[top=.5in, bottom=.5in, left=.5in, right=.5in]{geometry}
\usepackage{textcomp}

\begin{document}
\setlength\columnsep{-30pt}
%\setlength{\multicolsep}{6.0pt plus 2.0pt minus 1.5pt}% 50% of original values
\name{Connor F. Henley}
%@replace
\email{} % school email
\phone{} % personal email
\address{} % address line one
\address{} % address line two
\website{https://connorhenley.engineer}
\website{} % phone number
%@replend

\begin{resume}
 \setlength
 \multicolsep{2pt}

\section{OBJECTIVE:} % CAREER OBJECTIVE for real resume
	To be selected as a Resident Advisor, which will allow me to assist residents with their transition to college life and help them succeed during their time at RIT.
\vspace{-2pt}
\section{EDUCATION:} 
	\textbf{Rochester Institute of Technology} \hfill Rochester, NY \\
	Master of Science in Computer Engineering \hfill Expected May 2021 \\
	Bachelor of Science in Computer Engineering, Minor in Computer Science \hfill Expected May 2021\\ 
	GPA: 3.46
\vspace{-2pt}

\section{COURSES:}
		\begin{multicols}{2}
		\setlength\columnsep{1pt}
			\begin{itemize}
				\setlength{\itemindent}{-25pt}
				%\itemsep -2pt
				%\item[] example entry
				\item[] Assembly Language Programming
				\item[] Circuits I
				\item[] Computer Organization
				\item[] Digital Systems Design I \& II
				\item[] \hspace{12.5pt} Introduction to Software Engineering
				\item[] \hspace{12.5pt} Mechanics of Programming
				\item[] \hspace{12.5pt} Multi-variable Calculus
				\item[] \hspace{12.5pt} University Physics I \& II
			\end{itemize}
		\end{multicols}
\vspace{-2pt}
\section{SKILLS:}

	\textbf{Software:} Arduino IDE, Cura, Git, Linux, Microsoft Office, Windows \\
	\textbf{Programming Languages:} ARM Assembly, C, Java, \LaTeX, Python 3, VHDL 
	%\textbf{Hardware:} 3D Printer, Multimeter, Oscilloscope, Signal Generator, Soldering

\section{EXPERIENCE:}

	FIRST Robotics \hfill Oct 2009 - Present
	\begin{itemize}
		\item Volunteered at events to keep the event running smoothly \hfill Apr 2014 - Present
		\item Co-captain on FRC (FIRST Robotics Competition) Team 1699 \hfill Sept 2012 - May 2016
		\item Directed the local FLL (FIRST Lego League) team \hfill Oct 2009 - Dec 2011
	\end{itemize}
	\vspace{-8pt}

	Grader for Department of Computer Science \hfill Aug 2017 - Present\\
	Rochester Institute of Technology, Rochester NY
	\begin{itemize}
		\item Graded Mechanics of Programming programs (written in C) of students in a Linux environment
		\item Collaborated with another grader and the professor to ensure fair grading of all students
	\end{itemize}
	\vspace{-8pt}	

	Public Relations for RIT FIRST \hfill May 2017 - Present\\
	Rochester NY
	\begin{itemize}
		\item Coordinated events and mentor-ship for FIRST teams near RIT
		\item Worked with other members of the Executive Board to solve issues in the club
	\end{itemize}
	%\vspace{-8pt}

\section{PROJECTS \\AND LABS:}

	%{\sl ini-reader and ini-editor for Team 1699}
	%	\begin{itemize}
	%		\itemsep 0pt
	%		\item Status: \textit{Completed, Updated Often}
	%		\item Rewrote code to read configuration files stored on the robot. The original project was written in LabView, but it was rewritten in Java, and features were added in the process.
	%		\item The code has been used on the 2016 and 2017 competition robots, and in the swerve and autonomous projects.
	%		\item The ini-editor was designed to make the editing of configuration files on the robot easier, and the simulator was written to show how a configuration file would be interpreted.
	%		\item The editor was written in Python with Tk and the simulator was written in Java with JavaFX, and all code can be found on the Team 1699 GitHub.
	%	\end{itemize}

	Arduino Clock
	\begin{itemize}
		\item Created a internet-enabled clock using an Arduino Uno, TFT screen, Ethernet module, C++ code, and a custom 3D printed case
	\end{itemize}
	\vspace{-10pt}

	RIT FIRST's ImagineRIT Project
	\begin{itemize}
		\item Collaborated with other engineers and programmers to build small robots that are driven by attendees of ImagineRIT 
		\item Led the redesign of the project for ImagineRIT 2018
	\end{itemize}
	\vspace{-10pt}

	Swerve Code for FRC Team 1699
	\begin{itemize} 
		\item Programmed a swerve drive-train (and accompanying libraries), which is a complex drive base where all wheels can turn 360\textdegree \hspace{.5pt} and are independently steered and driven.
		%\item PID loops were used in order to make motions smoother and to automatically adjust for drift.
	\end{itemize}
	\vspace{-10pt}

	WebCheckers for Intro to Software Engineering
	\begin{itemize}
		\item Worked in a team of students to complete a web application for playing checkers (including an AI and game spectating)
		\item Utilized Java, git, Spark micro-webframework, Apache Maven, and JUnit
	\end{itemize}
	%\vspace{-10pt}

%\section{EXPERIENCE:}
%
%	Grader for Department of Computer Science \hfill Aug 2017 - Present\\
%	Rochester Institute of Technology, Rochester NY
%	\begin{itemize}
%		\item Graded Mechanics of Programming programs (written in C) of students in a Linux environment
%		\item Collaborated with another grader and the professor to ensure fair grading of all students
%	\end{itemize}
%
%	FIRST Robotics \hfill Oct 2009 - Present\\
%	Colchester CT and around Rochester NY
%	\begin{itemize}
%		\item Co-captain on FRC Team 1699 \hfill Sept 2012 - Apr 2016
%		\item Directed the local FIRST Lego League (FLL) team \hfill Oct 2009 - Dec 2011
%	\end{itemize}
%
%	Public Relations for RIT FIRST \hfill May 2017 - Present\\
%	Rochester NY
%	\begin{itemize}
%		\item Coordinated events and mentorship for FIRST teams near RIT
%		\item Worked with other members of the Execuitive Board to ensure that everything ran smoothly
%	\end{itemize}

\section{AWARDS:}
	\begin{itemize}
		\setlength{\itemindent}{-15pt}
		%\item A+ and Network Pro Certified
		\item Chairman's Award from FRC Team 1699
		\item Dean's List Nominee from FRC Team 1699
		\item Dean's List from Kate Gleason College of Engineering, Spring 2017
		%\item Honors from Bacon Academy
	\end{itemize}


\section{ACTIVITIES AND HOBBIES:}

%	FIRST Robotics \hfill Oct 2009 - Present
%	\begin{itemize}
%		\item Volunteered at events to keep the event running smoothly \hfill Apr 2014 - Present
%		\item Elected to co-captain position on FRC Team 1699 \hfill Sept 2012 - May 2016
%		\item Directed the local FLL (FIRST Lego League) team \hfill Oct 2009 - Dec 2011
%	\end{itemize}

	RIT FIRST Robotics Club\hfill Sept 2016 - Present
	\begin{itemize}
		\item Mentored local FIRST robotics teams, assisting with programming, electronics, team building, and game analysis
		\item Elected for the Executive Board in the Public Relations position \hfill May 2016 - Present
	\end{itemize}
\vspace{-5pt}

	3D Printing Enthusiast \\
	2017 Freshman Move-in Volunteer \\
	BrickHack 3 Hacker \hfill Feb 11 and 12 2017

%\section{COMMUNITY SERVICE:}  
%
%	Mentoring FIRST Robotics Teams \hfill Sept 2016 - Present
%	\begin{itemize}
%		\item Mentored FRC Team 1699 in Colchester CT and FLL (FIRST Lego League) 6149 in Rochester NY
%	\end{itemize}
%
%	Volunteering at FIRST Events \hfill Apr 2014 - Present
%	\begin{itemize}
%		\item Worked in a team in order to keep the event running and on-schedule
%		\item Volunteered at multiple FIRST events doing field reset, scoring and field setup/takedown
%	\end{itemize}
%
%\section{CLUBS AND HOBBIES:}
%	\begin{itemize}
%		\item BrickHack 3 Hacker
%		\item 3D Printing Enthusiast 
%	\end{itemize}
%
\end{resume}

\end{document}