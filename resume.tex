% resume.tex
%
% My resume
%
% @author Connor Henley, @thatging3rkid
% @author Chris Culpepper, @cculpapper
% @note see res.cls for more authors

% LaTeX resume using res.cls
\documentclass[line,margin]{res} 
%\usepackage{helvetica} % uses helvetica postscript font (download helvetica.sty)
%\usepackage{newcent}   % uses new century schoolbook postscript font 
\usepackage{multicol}
%\usepackage[top=.5in, bottom=.5in, left=.5in, right=.5in]{geometry}
\usepackage{textcomp}

\begin{document}
\setlength\columnsep{-30pt}
%\setlength{\multicolsep}{6.0pt plus 2.0pt minus 1.5pt}% 50% of original values
\name{Connor Henley}
\email{} % school email
\phone{} % personal email
\address{} % address line one
\address{} % address line two
\website{https://connorhenley.engineer}
\website{} % phone number
 
\begin{resume}
 \setlength
 \multicolsep{2pt}

\section{OBJECTIVE:} % CAREER OBJECTIVE for real resume
	%To be selected as a Resident Advisor, which will allow me to assist residents with their transition to college life and help them succeed during their time at RIT.
	To obtain an co-op in the Computer Engineering field, which will allow me to gain experience in the field and to apply topics learned in class.

\section{EDUCATION:} 
	Rochester Institute of Technology \hfill Rochester, NY \\
	Bachelor of Science in Computer Engineering \hfill Expected May 2021\\ 
	GPA: 3.37

\section{COURSES:}
		\begin{multicols}{2}
		\setlength\columnsep{1pt}
			\begin{itemize}
				\setlength{\itemindent}{-25pt}
				%\itemsep -2pt
				%\item[] example entry
				\item[] Assembly Language Programming
				\item[] Calculus I \& II
				\item[] Computer Science I \& II
				\item[] Digital Systems Design I
				\item[] \hspace{12.5pt} Introduction to Software Engineering
				\item[] \hspace{12.5pt} Mechanics of Programming
				\item[] \hspace{12.5pt} Multivariable Calculus
				\item[] \hspace{12.5pt} University Physics I \& II
			\end{itemize}
		\end{multicols}

\section{SKILLS:}

	\textbf{Software:} Altera Quartus, Arduino IDE, Cura, Git, Linux, Microsoft Office, Solidworks, Windows \\
	\textbf{Programming Languages:} ARM Assembly, C, Java, LabView, \LaTeX, Python, VHDL \\
	\textbf{Hardware:} Multimeter, Oscilloscope, Signal Generator, Soldering 

% 		\begin{multicols}{2}
%			\begin{itemize}
%				\setlength{\itemindent}{-25pt}
%				%\itemsep -2pt
%				%\item[] example entry
%				\item[] ARM Assembly
%				\item[] C
%				\item[] Java
%				\item[] Microsoft Office
%				\item[] \hspace{12.5pt} Python
%				%\item[] nginx and WordPress
%				\item[] \hspace{12.5pt} \LaTeX
%				\item[] \hspace{12.5pt} VHDL
%				\item[] \hspace{12.5pt} Windows and Linux
%			\end{itemize}
%		\end{multicols}

\section{PROJECTS \\AND LABS:}

	%{\sl ini-reader and ini-editor for Team 1699}
	%	\begin{itemize}
	%		\itemsep 0pt
	%		\item Status: \textit{Completed, Updated Often}
	%		\item Rewrote code to read configuration files stored on the robot. The original project was written in LabView, but it was rewritten in Java, and features were added in the process.
	%		\item The code has been used on the 2016 and 2017 competition robots, and in the swerve and autonomous projects.
	%		\item The ini-editor was designed to make the editing of configuration files on the robot easier, and the simulator was written to show how a configuration file would be interpreted.
	%		\item The editor was written in Python with Tk and the simulator was written in Java with JavaFX, and all code can be found on the Team 1699 GitHub.
	%	\end{itemize}

	Arduino Clock
	\begin{itemize}
		\item Created a internet-enabled clock using an Arduino Uno, TFT screen, Ethernet module, C++ code, and a custom 3D printed case
	\end{itemize}

	Delta Editor
	\begin{itemize}
		\item Designed and programmed a Unix text editor with basic functionality (open, save, scrolling, status bar, etc) in C using ncurses
	\end{itemize}

	RIT FIRST's ImagineRIT Project
	\begin{itemize}
		\item Collaborated with other engineers and programmers to build small robots that are driven by attendees of ImagineRIT 
		\item Led the redesign of the project for ImagineRIT 2018
	\end{itemize}

	Swerve Code for FRC (FIRST Robotics Competition) Team 1699
	\begin{itemize} 
		\item Programmed a swerve drivetrain (and accompanying libraries), which is a complex drivetrain where all wheels can turn 360\textdegree and are independently steered and driven.
		%\item PID loops were used in order to make motions smoother and to automatically adjust for drift.
	\end{itemize}

\section{EXPERIENCE:}

	Grader for Department of Computer Science \hfill Aug 2017 - Present\\
	RIT, Rochester NY
	\begin{itemize}
		\item Graded C programs of 40 students in a Linux environment
		\item Worked in a team in order to ensure fair grading of all students
	\end{itemize}

%	FIRST Robotics \hfill Oct 2009 - Present\\
%	Colchester CT and around Rochester NY
%	\begin{itemize}
%		\item Co-captain on FRC Team 1699 \hfill Sept 2012 - Apr 2016
%		\item Directed the local FIRST Lego League (FLL) team \hfill Oct 2009 - Dec 2011
%	\end{itemize}
%
%	Public Relations for RIT FIRST \hfill May 2017 - Present\\
%	Rochester NY
%	\begin{itemize}
%		\item Coordinated events and mentorship for FIRST teams near RIT
%		\item Worked with other members of the Execuitive Board to ensure that everything ran smoothly
%	\end{itemize}

\section{AWARDS:}
	\begin{itemize}
		\setlength{\itemindent}{-15pt}
		\item A+ and Network Pro Certified
		\item Chairman's Award from FRC Team 1699
		\item Dean's List Nominee from FRC Team 1699
		\item Dean's List from Kate Gleason College of Engineering, Spring 2017
		%\item Honors from Bacon Academy
	\end{itemize}


\section{ACTIVITIES AND HOBBIES:}

	FIRST Robotics \hfill Oct 2009 - Present
	\begin{itemize}
		\item Volunteered at events to keep the event running smooth \hfill Apr 2014 - Present
		\item Elected for co-captain on FRC Team 1699 \hfill Sept 2012 - May 2016
		\item Directed the local FLL (FIRST Lego League) team \hfill Oct 2009 - Dec 2011
	\end{itemize}

	RIT FIRST \hfill Sept 2016 - Present
	\begin{itemize}
		\item Mentor local FLL and FRC teams, assisting with programming, electronics, team building, and game analysis
		\item Elected for the Executive Board in the Public Relations position \hfill May 2016 - Present
	\end{itemize}

	3D Printing Enthusiast \\
	BrickHack 3 Hacker

%\section{COMMUNITY SERVICE:}  
%
%	Mentoring FIRST Robotics Teams \hfill Sept 2016 - Present
%	\begin{itemize}
%		\item Mentored FRC Team 1699 in Colchester CT and FLL (FIRST Lego League) 6149 in Rochester NY
%	\end{itemize}
%
%	Volunteering at FIRST Events \hfill Apr 2014 - Present
%	\begin{itemize}
%		\item Worked in a team in order to keep the event running and on-schedule
%		\item Volunteered at multiple FIRST events doing field reset, scoring and field setup/takedown
%	\end{itemize}
%
%\section{CLUBS AND HOBBIES:}
%	\begin{itemize}
%		\item BrickHack 3 Hacker
%		\item 3D Printing Enthusiast 
%	\end{itemize}
%
\end{resume}

\end{document}