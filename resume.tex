% resume.tex
%
% My resume
%
% @author Connor Henley, @thatging3rkid
% @author Chris Culpepper, @cculpapper
% @note see res.cls for more authors

% LaTeX resume using res.cls
\documentclass[line,margin]{res} 
\usepackage[utf8]{inputenc} % set the input encoding to UTF-8
%\usepackage{helvetica} % uses helvetica postscript font (download helvetica.sty)
%\usepackage{newcent}   % uses new century schoolbook postscript font
\usepackage{multicol}
%\usepackage[top=.5in, bottom=.5in, left=.5in, right=.5in]{geometry}
\usepackage{textcomp}
\usepackage{siunitx}
\usepackage{hyperref} % should go last, does URLs

% force PDF size to letter
\pdfpagewidth 8.5in
\pdfpageheight 11in

% setup URLs to be blue colored with no box
\hypersetup{
	colorlinks=true,
	urlcolor=cyan
}

% offset constants
\newcommand{\SECTIONOFFSET}{-3.6pt}
\newcommand{\ITEMOFFSET}{-8pt}

\begin{document}
\setlength\columnsep{-30pt}
%\setlength{\multicolsep}{6.0pt plus 2.0pt minus 1.5pt}% 50% of original values
\name{Connor F. Henley}
%@replace
\email{} % school email
\phone{} % personal email
\address{} % address line one
\address{} % address line two
\website{\url{https://cfh.sh}}
\website{} % phone number
%@replend

\begin{resume}
	 \setlength
	 \multicolsep{2pt}

    \vspace{-10pt} % top offset

	\section{OBJECTIVE:} % CAREER OBJECTIVE for real resume
		Dedicated Computer Engineering graduate looking to start a career in low-level software or high-level hardware (FPGA), where I can apply topics learned in class and experience from co-ops.

		\vspace{\SECTIONOFFSET}

	\section{EDUCATION:} 
		
		\textbf{Rochester Institute of Technology} \hfill Rochester, NY\\
		Master of Science in Computer Engineering \hfill Expected May 2021

		\begin{multicols}{2}
			\setlength\columnsep{1pt}
			\begin{itemize}
				\setlength{\itemindent}{-10pt}
				\item[] Compiler Construction
				\item[] Hardware/Software Design for Crypto. Apps.
				\item[] \hspace{16pt} Real-time \& Embedded Systems
				\item[] \hspace{16pt} Reconfigurable Computing
			\end{itemize}
		\end{multicols}

		\vspace{-12pt}
		Thesis Title: \textit{Hybrid Scheduler for Performant High-Level Synthesis} (in-progress) \\
		Thesis Advisor: Dr. Sonia L\'opez Alarc\'on 
		\vspace{\ITEMOFFSET}

		Bachelor of Science in Computer Engineering \hfill Expected May 2021\\ 
		Minor in Computer Science, Immersion in ASL and Deaf Cultural Studies \hfill GPA: 3.70

		\begin{multicols}{2}
			\setlength\columnsep{1pt}
			\begin{itemize}
				\setlength{\itemindent}{-10pt}
				\item[] Computer Organization \& Architecture
				\item[] Digital Systems Design I \& II
				\item[] \hspace{16pt} Interface \& Digital Electronics
				\item[] \hspace{16pt} Operating Systems
			\end{itemize}
		\end{multicols}

		\vspace{\SECTIONOFFSET}

	\section{SKILLS:}

		\textbf{Software:} FreeRTOS, git, Keil \si{\micro}\hspace{-.5pt}Vision, Linux, STM32CubeIDE, Xilinx SDK \& Vivado\\
		\textbf{Programming Languages:} ARM Assembly, C, C++, C\#, Java, \LaTeX, Matlab, Python 3, VHDL \\
		\textbf{Hardware:} 3D Printer, Multimeter, Oscilloscope, Power Supply, Signal Generator, Soldering

		\vspace{\SECTIONOFFSET}

	\section{PROJECTS \\AND LABS:}

		%Arduino Clock
		%\begin{itemize}
		%	\item Created an internet-enabled clock using a microcontroller, TFT screen, Ethernet module, C++ code, and a custom 3D printed case
		%	%\item Working on a redesign that will use a different display technology, improved Ethernet module, and integrate a real-time clock
		%\end{itemize}
		%\vspace{\ITEMOFFSET}

		%Flour Programming Language
		%\begin{itemize} 
		%	\item Designed and implemented (the lexer, tokenizer and preprocessor are functional) the top-half of a compiler for a C-with-classes programming language, which will generate LLVM IR
		%\end{itemize}
		%\vspace{\ITEMOFFSET}

		NXP Cup Car for Interface \& Digital Electronics
		\begin{itemize}
			\item Wrote software to autonomously drive a model car around a track as fast as possible
			\item Interfaced with various sensors and used different control theory methodologies
		\end{itemize}
		\vspace{\ITEMOFFSET}

		Power Wheels Universal Control Interface
		\begin{itemize}
			\item Worked in multidisciplinary team of students to build modified Power Wheels vehicle, used to help children with limited mobility learn how to drive wheelchairs
			\item Developed code-base that processed analog joystick input and generated analog output
		\end{itemize}
		\vspace{\ITEMOFFSET}

		RIT FIRST ImagineBots
		\begin{itemize}
			\item Led a project to redesign our table-top robots, which are driven by attendees of ImagineRIT 
			\item Collaborated with other students to design the architecture, write firmware and application software, as well as design some of the circuitry used in the control system
		\end{itemize}

		\vspace{\SECTIONOFFSET}

	\section{EXPERIENCE:}

		Senior Project Computer Engineer \hfill 2021 - Present (Incoming)\\
		Lutron Electronics, Boston MA
		\vspace{\ITEMOFFSET}

		Computer Engineering Firmware Co-op \hfill Aug 2018 - Dec 2018, May 2019 - Aug 2019\\
		Diebold Nixdorf, North Canton OH
		\begin{itemize}
			%\item Worked with the Firmware team of the Module and Systems Engineering department on debugging tools and automated firmware testing.
			\item Maintained and improved a firmware debugging tool which was used by team members daily
			\item Wrote automated firmware tests in C\# and participated in Agile software development
		\end{itemize}
		\vspace{\ITEMOFFSET}

		Teaching Assistant and Grader\\
		Rochester Institute of Technology, Rochester NY
		\begin{itemize}
			\item Graded C programs of students in a Linux environment \hfill Aug 2017 - May 2018
			\item Graded Java projects using IntelliJ and Java 12 \hfill Sept 2019 - Dec 2019
			\item Graded and helped student write and debug VHDL using Vivado \hfill Aug 2020 - Dec 2020
			%\item Collaborated with other student workers and the professor to ensure fair grading of all students
		\end{itemize}

		\vspace{\SECTIONOFFSET}

	\section{AWARDS:}

		A+ and Network Pro Certified \\
		Chairmans's Award and Dean's List Nominee from FRC Team 1699 \\
		Dean's List from Kate Gleason College of Engineering, Spring 2017 through Fall 2020
		
		\vspace{\SECTIONOFFSET}

	\section{ACTIVITIES AND HOBBIES:}

		FIRST Robotics \hfill Oct 2009 - Present
		\begin{itemize}
			\item Assisted teams with their robot as a Control System Advisor \hfill Jun 2018 - Present
			\item Volunteered at FIRST events to keep the event running smoothly \hfill Apr 2014 - Present
			%\item Elected to co-captain position on FRC Team 1699 \hfill Sept 2012 - May 2016
			%\item Directed the local FLL (FIRST Lego League) team \hfill Oct 2009 - Dec 2011
			\item Participated on local FIRST teams, including leadership roles \hfill Oct 2009 - May 2016
		\end{itemize}
		\vspace{\ITEMOFFSET}

		RIT FIRST Alumni Association\hfill Sept 2016 - Present
		\begin{itemize}
			\item Mentored various local FIRST robotics teams (FRC Team 3838 and local FLL teams), assisting with programming, electronics, team building, and game analysis
			\item Led the club as the President, ensuring club projects were successful \hfill May 2019 - Present
		\end{itemize}
		\vspace{\ITEMOFFSET}

		3D Printing Enthusiast,  Amateur Radio Technician
		%BrickHack 3 Hacker \hfill Feb 11 and 12 2017
		
\end{resume}
\end{document}
